\documentclass[a4paper,11pt]{article}
\usepackage[brazil]{babel}
\usepackage[utf8]{inputenc}

\usepackage{verbatim}

\title{\normalfont{Enunciado - Problema da Conexidadade 2D}}
\date{\today}

\begin{document}
\maketitle % Print the title/author/date block

\setcounter{tocdepth}{2} % Set the depth of the table of contents to show sections and subsections only

%\tableofcontents % Print the table of contents
%\listoffigures % Print the list of figures
%\listoftables % Print the list of tables

\section{Enunciado}
Suponha que $P_1,\dots,P_n$ sejam pontos no quadrado unitário $Q = [0,1]^2$ e seja $d$ um número real positivo. Declaramos $P_i$ e $P_j$ conexos se a distância entre esses pontos é menor ou igual a $d$. Considere agora o fecho transitivo dessa relação. Dizemos que a configuração de pontos $P_1,\dots,P_n$ é $d$-conexa se essa relação de equivalência tem apenas uma classe de equivalência. \\
O problema computacional que queremos resolver é como segue. \\
PC2D: Problema da conexidade 2D

\begin{itemize}
    \item Entrada: $P_1,\dots,P_n$ ($n\geq1$) pontos no quadrado unitário $Q$ e um número real $d>0$.
    \item Problema: determine se $P_1,\dots,P_n$ forma uma configuração $d$-conexa.
\end{itemize}

Escreva um programa que resolve PC2D. Seu programa será testado para valores grandes de $n$ e valores pequenos de $d$.

Sugestão: veja Web Exercise 1.3.45 (Gridding) de Algs4.

\section{Relatório}
É necessário entregar, também, um relatório sobre a sua implementação do problema. Este faz parte da avaliação.

\section{Especificações}
Você deve entregar um arquivo chamado PC2D.java que implementa uma classe chama PC2D. Essa classe deve ter uma main que executa o seguinte procedimento: \\
Recebe, da entrada padrão, um inteiro $n$ e um real $d$ seguidos de $n$ pontos dados na forma $x_i$, $y_i$ e imprime, na saída padrão \texttt{Sim}, caso a relação descrita tenha apenas uma classe de equivalência e \texttt{Nao} (sem acento) caso contrário. 

\section{Exemplos}
\begin{samepage}
\subsection*{Exemplo 1}
\bf{Entrada}
\verbatiminput{in1}
\bf{Saída} (referente à Entrada)
\verbatiminput{out1}
\end{samepage}
\begin{samepage}
\subsection*{Exemplo 2}
\bf{Entrada}
\verbatiminput{in2}
\bf{Saída} (referente à Entrada)
\verbatiminput{out2}
\end{samepage}


%\bf{Entrada 1}
%\exemplo{./in1.txt}
%\bf{Saída 1} (referente à Entrada 1) \\
%\exemplo{out1.txt


\end{document}
